\section{Particle In Cell}
% Making it a charged simluation
% Adding the particle in cell solvers
\subsection{Code Flow}
% How this changes the code flow from before
\subsection{Equations}

particles move according to the Lorenz force, seen in Equation ref{eqn:lorenz}.


% LORENZ EQUATION


However, there can be patches of charge inequality. The characteristic length of those charge inequalities can be calculated through the Debye Length, \(\lambda_{De}\), shown in Equation \ref{eqn:debye} \cite{debye}. \par


\Needspace{8\baselineskip}
\begin{equation}
    \label{eqn:debye}
    \lambda_{De} = \sqrt{\frac{\epsilon_0 \: T_e}{e \: n_0}}
\end{equation}
\(T_e\) = Electron Temperature \\
\(\epsilon_0\) = Permittivity of free space \\
\(e\) = elementary charge \\
\(n_0\) = plasma density \par

\subsection{Finite Volume}
\subsection{Execution time}
