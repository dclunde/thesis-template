
The goal of this thesis is three-fold. The first goal is to help develop a homegrown DSMC simulation tool for Cal Poly. The second goal is to implement charged particles into the DSMC to create a hybrid DSMC-PIC simulation. The third goal is to design the code architecture and systems to have both a low probability of becoming unusable and a similarity to the upcoming homegrown fluid simulation. These are all intermediary goals which work towards the final end of creating a full simulation of an electric engine. This thesis shows that the three goals have been accomplished through the latest upgrade of the new homegrown code, SINATRA. \par

\indent Building a homegrown CFD code is a complex task. The early developers have worked together to build a robust and flexible DSMC code base. This has been demonstrated through Alliston and Galvez's theses \cite{Galvez2018a,mac_thesis}. It has the ability to model many different flow conditions, species, and boundary conditions. Its execution time (i.e. under 2 minutes for 2 million particles) allows for accurate simulations in an institution research setting. It has been validated across its various collision and sphere models. \par


\indent In order to eventually model a full electric thruster, this part of the simulation must be able to model charged particles for the plume. This has been achieved by making SINATRA a DSMC-PIC hybrid. A simple and robust method is used for discretizing Poisson's Equation: The Finite Difference method. This involves creating a 7-point 3D stencil, Neumann boundary conditions, slip conditions through averaging. For solving the resulting set of linear equations, the Gauss-Seidel iterative method was chosen. The implementation in SINATRA was verified against a verified solver and shown to converge on the 3D stencil. The accuracy of the PIC implementation has been validated through two important test cases, ambipolar diffusion and steady state flow. In the ambipolar diffusion test case, it is clear that the charged particles create a force which greatly reduces diffusion so that within \(5\times 10^{-5}\) seconds only 0.071\% of the particles leave the domain. With the steady state flow, the potential was seen to quickly assume a steady state solution and change slowly as the particles moved through the domain. \par

\indent The code base has been converted from a small developer's program to a university standard code. It is managed through GitHub\textsuperscript{\textregistered} in order to be used by multiple university developers. It has been upgraded with a user manual, consistent and simple documentation, and a simplified workflow. Two large bottlenecks, particle pushing and particle linking, have been greatly mitigated through parallization and increased data throughput so that timely simulations may be made. In addition, a distribution method has been created with easy-to-use executables and a GUI for user control. \par

\indent There are still many ways to improve upon SINATRA. Those are discussed in Section \ref{sec:future}. However, SINATRA is a strong first iteration of a university level DSMC-PIC code base. This work will help SINATRA reach its goals and Cal Poly will enabled to do novel research in modeling full electric thrusters.
