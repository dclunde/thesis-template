
The goal of this thesis is three fold. Firstly, to help develop a homegrown DSMC simulation for Cal Poly. Secondly, to implement charged particles into the DSMC to create a hybrid DSMC-PIC simulation. And finally, to design the code architecture and systems to both have a low probability of becoming unusable and to be similar to the upcoming homegrown fluid simulation. These are all smaller goals towards the final goal of creating a full simulation of an electric engine. This thesis shows that those goals have been accomplished in the latest upgrade of the new homegrown code, SINATRA. \par

\indent Building a homegrown CFD code is a complex task. The early developers have worked together to build a robust and flexible DSMC code base. This has been demonstrated through Alliston and Galvez's theses \cite{mac_thesis} \cite{Galvez2018a}. It has the ability to model many different flow conditions, species, and boundary conditions. It's execution time, for example under 2 minutes for 2 million particles, allow for accurate simulations in a institution research setting. It has been validated across its various collision and sphere models. \par


\indent In order to eventually model a full electric thruster, this part of the simulation must be able to model charged particles for the plume. This has been achieved by making SINATRA a DSMC-PIC hybrid. A simple and robust method was used for discretizing Poisson's Equation, the Finite Difference method, as well as for solving the resulting set of linear equations, the Gauss-Seidel iterative method. The accuracy has been validated through two important test cases, Ambipolar diffusion and Collette flow. \par

\indent The code base has been converted from a small developers platform to a university standard code. It is managed through GitHub\textsuperscript{\textregistered} in order to be managed by multiple university developers. It has been upgraded with a user manual, consistent and simple documentation, and a simplified workflow. Two large bottlenecks, particle pushing and particle linking, have been greatly reduced through parallization and increased data throughput so that timely simulations may be made. And a distribution method has been created with easy to use executables and a GUI for control. \par

\indent There are still many ways to improve upon SINATRA. Those are discussed in section \ref{sec:future}. However, SINATRA is a strong first iteration of a university level DSMC-PIC code base. This work will help SINATRA reach it's goals and Cal Poly will able to do novel research with modeling full electric thrusters. From this humans could find new and exciting ways to explore our Universe.
