\section{Motivation}
There are many methods to create something new. They all begin with an idea which needs realization. The path to realization in some cases can involve simply building the final project. However, with Aerospace Engineering that is not the case. To achieve the end goal much planning is needed. Part of that planning includes 

\subsection{Overarching Goal}
This thesis is one project in a planned series of projects. The goal is to have a Cal Poly homegrown simulation which can simulate an entire electric thruster. This is a task which has not yet been accomplished in manner which is reasonable for university researchers to use. A simulation of a full electric thruster requires a fluid based simulation for the gas inside the thruster and then a rarefied gas code for the exhaust plume. There aren't codes which have been built as one simulation, instead researcher attempt to join a fluid code and rarefied gas code together. This is where Cal Poly's Aerospace department is unique. A fluid code and a rarefied gas code are being developed from the ground up with a focus on the similarities. The early developers and advisors work together to build these codes to be compatible from the beginning. \par

\indent This thesis is the third project of the SINATRA code which will be able to simulate the thruster's plasma exhaust plume. The first three developers worked together in a staggered capacity to build SINATRA up to a working DSMC-PIC code. The first, David Galvez, developed the base framework and kinematics \cite{Galvez2018a}. Next Robert Alliston built up the collisions and particle models cite{Mac's Thesis}. This thesis adds charged particle simulation to SINATRA. At the same time as this thesis, Anthony Gay is building the fluid side simulation. Intentionally many things will be common between the codes. They are both written in C++ and are class based. They share the same process control items including the execution style, distribution method, and the other systems engineering items shown in Chapter \ref{chap:systems}. They share the same mesh type, input class, and other items to help them work together as one simulation. It is slated for future work for a project which takes both simulations and builds the interaction system to connect them across each timestep. 



% cite no full thruster code
% Cite macs thesis when it comes in