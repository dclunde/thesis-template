\section{Motivation}
There are many methods to create something new and they all begin with an idea that needs realization. The path to realization in some cases can involve simply building the final project; however, with Aerospace Engineering that is not the case. In order to achieve the end goal much planning is needed. A idea is formed, the feasibility is researched, it is designed, and then built. With many Aerospace projects, designing a cutting edge project requires accurate physical modeling in order confirm the feasibility of the design and examine the effects of iterations. Therefore, the field of Aerospace simulation and modeling is a large and extensive field. It is one which is constantly changing and expanding as computing power becomes exponentially stronger. The boom in computing power has opened the door to this thesis. \par

\indent A large part of Aerospace research takes place within low density states of matter. Atmospheric reentry of spacecraft, objects in Low Earth Orbit, planes flying at extremely high altitudes, and interactions with plasma all operate within in relativly low density. 
% knudsten number

\subsection{Overarching Goal}
This thesis is one project in a series of projects. The goal is to have a Cal Poly homegrown simulation that can simulate an entire electric thruster. This is a task that has not yet been accomplished in manner easily accessible to university researchers. A simulation of a full electric thruster requires a fluid based simulation for the gas inside the thruster and then a rarefied gas code for the exhaust plume. There are not codes that have been built as one simulation, instead researcher attempt to join a fluid code and rarefied gas code together. Cal Poly’s Aerospace department is unique in that a fluid code and a rarefied gas code are being developed from the ground up in the exact same manner. The early developers and advisors worked together to build these codes to be compatible from the beginning. \par

\indent This thesis is the third project of the SINATRA code which will be able to simulate the thruster’s plasma exhaust plume. The first three developers worked together in a staggered capacity to build SINATRA up to a working DSMC-PIC code. The first, David Galvez, developed the base framework and kinematics \cite{Galvez2018a}. Next Robert Alliston built up the collisions and particle models cite{Mac’s Thesis}. This thesis adds charged particle simulation to SINATRA. At the same time as this thesis, Anthony Gay is building the fluid side simulation. Intentionally many things will be common between the codes. They are both written in C++ and are class based. They share the same process control items including the execution style, distribution method, and the other systems engineering items shown in Chapter \ref{chap:systems}. They share the same mesh type, input class, and other items to help them work together as one simulation. A future project is slated that will take both simulations and build the interaction system to connect them across each timestep.
. 



% cite no full thruster code
% Cite macs thesis when it comes in