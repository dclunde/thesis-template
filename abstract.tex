Powering spacecraft with electric propulsion is becoming more common, especially in CubeSat-class satellites. On account of the risk spacecraft interactions, it is important to have robust analysis and modeling tools of electric propulsion engines, particularly of the plasma plume. The Navier-Stokes equations used in classic continuum computational fluid dynamics do not apply to the rarefied plasma, and therefore another method must be used to model the flow. A good solution is to use the Direct Simulation Monte Carlo (DSMC) method, which uses a combination of particle modeling and statistical methods for modeling the simulated molecules. 

A DSMC simulation known as SINATRA has been developed with the goal to model electric propulsion plumes. SINATRA uses an octree mesh, written in C++, and is designed to be expanded by further research. SINATRA has been initially validated through several tests and comparisons to theoretical data and other DSMC models. Multiple molecular model and collision schemes were also tested for validity, accuracy, and speed. This thesis examines expanding the functionality of SINATRA to simulate charged particles and make SINATRA a DSMC-PIC hybrid. The electric potential is calculated through a 7 point 3D stencil on the mesh nodes and solved with a Gauss-Seidel solver. It is validated through test cases of charged particles to demonstrate the accuracy and capabilities of the model. It includes additional features to simplify further research including  a comprehensive User Manual, industry version control, simple text file inputs, GUI control, and simple parallelism of the simulation. 

%[[The first paragraph of the abstract is good. The second one is more of a "methods" section. I want results! What did you find? 

%I should be able to read an abstract and understand the premise of an experiment (not the details) and the results. ]]