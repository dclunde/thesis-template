Powering spacecraft with electric propulsion is becoming more common, especially in CubeSat-class satellites. On account of the risk of spacecraft interactions, it is important to have robust analysis and modeling tools of electric propulsion engines, particularly of the plasma plume. The Navier-Stokes equations used in classic continuum computational fluid dynamics do not apply to the rarefied plasma, and therefore another method must be used to model the flow. A good solution is to use the DSMC method, which uses a combination of particle modeling and statistical methods for modeling the simulated molecules. 

A DSMC simulation known as SINATRA has been developed with the goal to model electric propulsion plumes. SINATRA uses an octree mesh, is written in C++, and is designed to be expanded by further research. SINATRA has been initially validated through several tests and comparisons to theoretical data and other DSMC models. This thesis examines expanding the functionality of SINATRA to simulate charged particles and make SINATRA a DSMC-PIC hybrid. The electric potential is calculated through a 7-point 3D stencil on the mesh nodes and solved with a Gauss-Seidel solver. It is validated through test cases of charged particles to demonstrate the accuracy and capabilities of the model. An ambipolar diffusion test case is compared to a neutral diffusion case and the electric field is shown to stabilize the diffusion rate. A steady state flow test case shows the simulation is able to stabilize and solve the electric potential for a plume-like scenario. It includes additional features to simplify further research including  a comprehensive user manual, industry-standard version control, text file inputs, GUI control, and simple parallelism of the simulation. Compilation and execution are standardized to be simple and platform independent to allow longevity of the code base. Finally, the execution bottlenecks of linking particles to cells and particle moving were removed to reduce the simulation time by 95\%. 
