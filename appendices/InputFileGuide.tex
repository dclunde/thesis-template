\chapter{Input File Guide}
\label{app:inputfileguide}
This is the Input File Guide. It can be found in the ReadMe for the resources file. It explains each command of the input file and how it works.

\begin{verbatim}
Input file Guide  
  
SINATRA reads input files line by line.   
It searches for key words and ignores words which don't fit them.   
Then it takes the items in the line under the keyword and
 parses that for the simulation.  
  
There are 6 sections to the input file. The sections are
 deliminated by *  
After the first section, the order of the sections does not matter.  
The order of the trigger words do not matter within the sections  
  
Section 1 - This is the intro section. It does not have
 a header trigger  
  
"Mesh Input Filename"                               - path name to 
 the file which holds the mesh information  
"Number of Real Particles to Simulation Particles"  - number as a 
 double which is the ratio of real particles per simulation particles  
"Boundary Conditions"                               - boundary
 conditions of the walls. 6 space deliminated string arguments
  for the type of the walls. Options are "INFLOW","OUTFLOW",
   "SWALL","DWALL","PWALL". Order of the walls is
    -X,+X,-Y,+Y,-Z,+Z.  
"Collision Scheme"                                  - an integer
 which signifies which collision scheme to use  
"Sphere Model"                                      - an integer
 which signifies which sphere model to use  
"Total Simulation Time"                             - a double
 showing the total time the simulation should run  
"Time Step"                                         - a double of
 the amount of time for each time step  
  
  
Section 2 - Trigger "Initial Conditions"   
This section contains simple initial conditions  
  
"Number Density"     - a double with the number density
 for the whole simulation  
"Mixture"            - a space separated line of alternating
 integers and doubles. The first number is the species
  id and the second is the percentage it is in the
   mixture being simulated  
"Stream Temperature" - a double temperature in Kelvin  
"Stream Velocity"    - space separated doubles which are 
 the X,Y,Z direction of the stream velocity  
  
  
Section 3 - Trigger "Boundary Condition Information"  
This section contains the information about the Boundaries  
  
There are 6 items which should be put in this. 
 They each start with "BC X".   
X is the number of the boundary from 0 to 5
 following -X,+X,-Y,+Y,-Z,+Z.   
Each section ends with a "&".   
They contain  
  
"Number Density"  
"Mixture"  
"Stream Temperature"  
"Stream Velocity"  
  
where each are defined the same way as in Section 2.  
  
  
Section 4 - Trigger "Output Information"  
This section defined the output variables needed to view the data  
  
"Output File Name"           - a string path to
 where the simple output file will be placed  
"Tecplot Base Name"          - a string path which
 is the base of the Tecplot output files. It will be
  edited with timestep information for each new file.
   WARNING - if the folder path does not exist, no
    files will be created.  
"Sample Cell Type"           - a flag for which sampling 
 should be used. 1 stands for leaf cells and 2 stands for 
  the immediate parents of the leaf cells  
"Tecplot Sample Frequency"   - an integer of the number of
 time steps between each sampling  
"Tecplot Sample Start Time"  - a double which defines the
 start time for Tecplot sampling  
"Particle Animation"         - a three lined argument.
 first is start time. second is frequency (integer), and
  final is the base file name for the animation files  
  
  
  
Section 5 - Trigger "Species Information"  
This section holds the information for the species to be used  
  
Each species section starts with the trigger "Species"  
After this a list of items are included which are added
 to that species class  
These are listed below in order.   
  
Reference Temperature            - Kelvin    - double  
Molecular Diameter               - meters    - double  
Mass                             - kilograms - double  
Rotational Degrees of Freedom    - number    - integer  
Vibrational Degrees of Freedom   - number    - integer  
Characteristic Temperature (Rot) - Kelvin    - double  
Characteristic Temperature (Vib) - Kelvin    - double  
Viscosity Index                  - number    - double  
Viscosity Coefficient            - number    - double  
Charge                           - Columbs   - double  
VSS Exponent (VSS sphere only)   - number    - double    
  
  
Section 6 - Trigger "Optional Keywords"  
Optional items for the simulation.   
These don't necessarily have a second line, they can be
 used just as single line triggers  
  
"diffusion"              - if trigger is there, enables 
 the diffusion test case  
"init_velocity_gradient" - Next line is the direction of 
 the gradient (X,Y,Z). Next line is 6 item deliminated by
  spaces with the inputs for the velocity gradient in the
   normal 6 direction order  
"Charged Simulation"     - triggers a charged simulation  
"Parallel Enabled"       - triggers a parallel simulation   
  
  
Any other questions about the input file can be solved
 by emailing the original developers. Or by reading 
  through input.cpp and input.h.  
  
\end{verbatim}