\chapter{Simple Distribution Walkthrough}
\label{app:walkthrough}
Seen below is a simple tutorial for starting with SINATRA. It can be found in the documentation/SimpleDistribution directory as ReadMe.md. 

\begin{verbatim}
SINATRA Simple Distribution Manual
Dominic Lunde


Hello! Welcome to a tutorial on how to use SINATRA.

First, requirements
1) Windows OS
2) Matlab or Matlab runtime - for output only

This will walk you through a simple simulation case. 
Then it will show you how to use the GUI and then you can have fun!



Step 1 - Mesh

We must create a mesh file. This is done through
 the Mesh.exe executable.
The MeshInput.txt file is already set up to create
 8 cells through 1 cut.
 
Therefore we can run the mesher by either:
a) Dragging MeshInput.txt onto Mesh.exe
b) Running "Mesh MeshInput.txt" into a
 command window set to this directory

After this is done, you should see your
 mesh file as a .plt file appear. Good!


Step 2 - Simulate

Now we can simulate some particles. 
Complex_Input.txt is set up to use
 8 cells and run 10 timesteps

Therefore we can run the mesher by either:
a) Dragging Complex_Input.txt onto DSMC.exe
b) Running "DSMC Complex_Input.txt"
 into a command window

All done! You will see a file called SINATRA_OUTPUT.txt.
This shows you very simple time step info. 
You will also see SINATRA_uniform_properties_000001.plt.
This has Tecplot data from the first time step.


Step 3 - Analysis

We want to see those particles move right? 
We need to change some things first.
Open Complex_Input.txt and go
 to the "Output Information" Section
Between the Tecplot Sample Frequency and the *** add this.

Particle Animation
0
1
VelocityOutput\velocityFromParticles_

The first line is the trigger word, next is the start time,
 then the frequency of output, and finally the base filename.
--Create the folder VelocityOutput 
 so these files have a place to go. 
Now run the simulation again. 

Now if you open VelocityOutput you will see 11 files! 
We can use those to view the particles and their movements.
--Open FlowVelocityDistributionAnim.m in Matlab and run

A file called SINATRA_gif.gif will be
 created and you can see your particles!
However, it's a mess and they don't seem to move. Let's change that. 
In Complex_Input 
--Change the Time Step to 1e-5 
--Change the Total Simulation Time to 10e-5.
This will allow the particles to move further each frame. 
Now let's make less particles.
--Set the Number Density to 1e-22.
Now rerun the simlulation and analysis
 and boom! Particles in a box. 

Step 4 - GUI

Now I'm going to let you play with the GUI. Open up SINATRA_GUI.exe. 
We're going to have to change a few things. 
SINATRA Output file - SINATRA_OUTPUT.txt
Tecplot Output      - SINATRA_uniform_properties.plt
Velocity Output     - VelocityOutput/velocityFromParticles_
Mesh File           - FE_NodeLocations_UnitCubeQuad8Cells_TEC.plt
Input File          - Complex_Input.txt

This GUI prints input files. 
Try it by pressing "Write File" then "Open File". 
Then change something and see how it changes the input file. 
Finally, you can press "Simulate" to run that input file on DSMC.exe.

Have fun!


\end{verbatim}