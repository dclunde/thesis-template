\chapter{Introduction}
\label{chap:intro}
\section{Motivation}
There are many methods to create something new and they all begin with an idea that needs realization. The path to realization in some cases can involve simply building the final project; however, with Aerospace Engineering that is not the case. In order to achieve the end goal much planning is needed. A idea is formed, the feasibility is researched, it is designed, and then built. With many Aerospace projects, designing a cutting edge project requires accurate physical modeling in order confirm the feasibility of the design and examine the effects of iterations. Therefore, the field of Aerospace simulation and modeling is a large and extensive field. It is one which is constantly changing and expanding as computing power becomes exponentially stronger. The boom in computing power has opened the door to this thesis. \par

\indent A large part of Aerospace research takes place within low density states of matter. Atmospheric reentry of spacecraft, objects in Low Earth Orbit, planes flying at extremely high altitudes, and interactions with plasma all operate within in relativly low density. 
% knudsten number

\subsection{Overarching Goal}
This thesis is one project in a series of projects. The goal is to have a Cal Poly homegrown simulation that can simulate an entire electric thruster. This is a task that has not yet been accomplished in manner easily accessible to university researchers. A simulation of a full electric thruster requires a fluid based simulation for the gas inside the thruster and then a rarefied gas code for the exhaust plume. There are not codes that have been built as one simulation, instead researcher attempt to join a fluid code and rarefied gas code together. Cal Poly’s Aerospace department is unique in that a fluid code and a rarefied gas code are being developed from the ground up in the exact same manner. The early developers and advisors worked together to build these codes to be compatible from the beginning. \par

\indent This thesis is the third project of the SINATRA code which will be able to simulate the thruster’s plasma exhaust plume. The first three developers worked together in a staggered capacity to build SINATRA up to a working DSMC-PIC code. The first, David Galvez, developed the base framework and kinematics \cite{Galvez2018a}. Next Robert Alliston built up the collisions and particle models cite{Mac’s Thesis}. This thesis adds charged particle simulation to SINATRA. At the same time as this thesis, Anthony Gay is building the fluid side simulation. Intentionally many things will be common between the codes. They are both written in C++ and are class based. They share the same process control items including the execution style, distribution method, and the other systems engineering items shown in Chapter \ref{chap:systems}. They share the same mesh type, input class, and other items to help them work together as one simulation. A future project is slated that will take both simulations and build the interaction system to connect them across each timestep.
. 



% cite no full thruster code
% Cite macs thesis when it comes in

\chapter{DSMC and SINATRA}
\label{chap:dsmc}
The DSMC method has been refined over years of implementation and testing. A large amount of that progress has been done by Bird, who introduced the DSMC method \cite{bird_76}. Much of the DSMC procedures and techniques discussed here have come from his book which walks a developer through the steps of creating a DSMC simulation \cite{bird_dsmc}. This chapter will explain the DSMC method and the SINATRA's implementation. \par

\section{DSMC Overview}

\indent At its core, the DSMC method is a particle pusher. It takes a domain, initializes particles in the domain, and, through each time-step, injects, moves, and collides particles. The simple breakdown of a DSMC flow can be seen in Figure \ref{fig:dsmc_flow}. The large difference between DSMC and a plain particle pusher is that each simulated particle is designed as a clump of actual particles. Therefore, each section and algorithm of the simulation must keep that in mind when calculating physical properties and events. However, for this thesis, a `super-particle' will be referred to as a particle for simplicity. \par

\begin{figure}
\centering
  \begin{tikzpicture}[node distance = 1cm, auto]
  \tikzstyle{block} = [rectangle, draw, fill=white, 
    text width=15em, text centered, rounded corners, minimum height=1em]
  \tikzstyle{line} = [draw, -latex']
    % Place nodes
        \node [block] (init) {Initialize System};
        \node [block, below of=init] (move) {Move Particles};
        \node [block, below of=move] (boundary) {Perform Boundary Interactions};
        \node [block, below of=boundary] (new) {Insert New Particles};
        \node [block, below of=new] (sort) {Sort Particles into Cells};
        \node [block, below of=sort] (collide) {Collide Particles};
        \node [block, below of=collide] (sample) {Sample Properties};
        \node [block, below of=sample] (check) {\(t \: < t_{final}\) \: ?};
        \node [block, below of=check] (stop) {Stop Simulation};
        % Draw edges
        \path [line] (init) -- (move);
        \path [line] (move) -- (boundary);
        \path [line] (boundary) -- (new);
        \path [line] (new) -- (sort);
        \path [line] (sort) -- (collide);
        \path [line] (collide) -- (sample);
        \path [line] (sample) -- (check);
        \path [line] (check) -- node {no} (stop);
        \draw (check) -- +(-10em,0) [-latex'] |- (move) node[left of=check,xshift=-6.5em,below] {yes};

    \end{tikzpicture}
    \caption[Basic DSMC flowchart]{Basic DSMC flowchart \cite{Galvez2018a}}
    \label{fig:dsmc_flow}
\end{figure}


\indent \textbf{Initialize System} There are two critical parts when initializing the simulation. First is the mesh structure. The simulation must know where each cell is in the domain, what shape the cell is, and what type of boundary each of its faces are. There are various algorithms and methods used in generating and storing this mesh. Those ones used for SINATRA are discussed in Section \ref{sec:octree}. The second important initialization is of the particles. Importantly, once initialized, the particles should have a uniformly random distribution in their position, as well as a normal distribution around the set domain velocity vector. There are two methods to initialize particles. The first is to randomly insert the number of simulated particles throughout the domain. However, this can cause problems with collision systems and with mesh-particle linking. Therefore, there is a second method called uniform distribution. In this method, the particles are distributed evenly to all of the cells and then within the cell they are randomly distributed. This is a valid method even though it removes some degree of randomness, because injecting and colliding particles re-establishes statistically random behaviour \cite{bird_dsmc,Galvez2018a,mac_thesis}. Therefore, accurate results can be gained in the uniformly distributed method as long as sufficient time-steps are included. The difference between the two systems can be seen in Figure \ref{fig:part_init}. Two smaller processes done during the Initialize section are reading information from the user input and linking the particles to the cells they are within. \par

\begin{figure}
    \centering
  \begin{minipage}[b]{0.49\textwidth}
    \includegraphics[width=\textwidth]{figures/psudo_init.png}
  \end{minipage} %
  \begin{minipage}[b]{0.49\textwidth}
    \includegraphics[width=\textwidth]{figures/uniform_init.png}

  \end{minipage}
  \caption[Uniform Particle Initialization]{Uniform Particle Initialization (left) The generic random initialization of particle position \cite{mac_thesis}. (right) The uniform distribution of particle position by each cell \cite{mac_thesis}.}
  \label{fig:part_init}
\end{figure}

\indent \textbf{Move Particles} This algorithm consists of going through each particle and updating its position using its velocity and the chosen time-step. This is a purely decoupled event and therefore can be split into parallel operations, which has been implemented and is shown in Section \ref{sec:execution}. \par

\indent \textbf{Perform Boundary Interactions} After moving a particle, it is checked to see if it exited its cell, and if so if it passed through a boundary on the way there. If so, the particle is processed through surface interactions. There are many types of surfaces that can be implemented in a DSMC simulation. A discussion of the ones implemented in this thesis are found in Section \ref{sec:models}. \par

\indent \textbf{Insert New Particles} In this stage particles are introduced to the domain through Inlets specified by the user. New particles are randomly inserted in terms of position and velocity, but are inserted with the same distributions required in initialization. \par

\indent \textbf{Sort Particles into Cells} Once the particles are all moved, all the new particles are added, the surface interactions are calculated, and the particles which are in new cells are linked back to those cells. This involves looking at nearby cells and determining within which cell the particle now lies, as well as removing particles no longer in the domain. \par

\indent \textbf{Collide Particles} Now collisions are calculated between the particles. This involves using the sphere model for the particle and choosing collision partners through a selection method. A discussion on the models available in SINATRA can be found in Section \ref{sec:models}. These collisions will change the velocity of the particles. \par

\indent \textbf{Sample Properties} Once this is completed, properties about the simulation can be sampled. Through sampling during the time-stepping loop, the analyst has the ability to view the simulation over time and determine if it has reached a steady state solution or view the transient or oscillatory nature of the fluid. This allows SINATRA to have the capacity to be a steady state or a transient simulation depending on the scenario and application. The method to sample properties in SINATRA can be found in Section \ref{sec:output}. \par

These steps continue until the simulation has completed the user specified time-steps. Then the simulation breaks from the loop, performs final data output, and ends the program. 





\section{SINATRA}


\subsection{Object Oriented}

\subsection{Octree Mesh}


\subsection{Code Flow}

\chapter{Systems Engineering}
\label{chap:systems}
All large aerospace projects can be viewed a group of integrated systems. The systems must be designed and managed in order to provide a reliable and efficient final product. Systems Engineering is the discipline which covers this area. SINATRA is a complex project and program which needs a smooth integration of the various program and sub-systems. This chapter will explain the various systems engineering implementations in SINATRA including documentation, work-flow, and user interfaces. 
\section{Documentation}
In order for SINATRA to be used by other users and developers, the code must be well-documented. Accordingly, systems of documentation are being put in place for all of the different levels of instructions, guidelines, comments, and information. The documentation  can be broken into two sections; for the user and for the developer. These documentation systems must be simple, reliable, clear, and resilient, in order to make the code easy to distribute, simple to learn to the level necessary, and for changes, bugs, and suggestions to be centralized.

\subsection{GitHub}
The system chosen to organize and host SINATRA for developers is GitHub\textsuperscript{\textregistered}. GitHub\textsuperscript{\textregistered} is an online file storage, syncing, and collaboration work space for developing code bases. It is popular and easy to use such that support for its use is strong. SINATRA is housed on GitHub\textsuperscript{\textregistered} by the author and is shared with Dr. Amelia Greig and Dr. David Marshall. On account of the code's license the repository is private and developers can get access by contacting the Aerospace Department at Cal Poly, SLO. \par
\indent GitHub\textsuperscript{\textregistered} has three major features used in SINATRA: the commits system, the branches system, and the ReadME files. The commits system is a method of allowing multiple developers to work on the same code base without breaking the other developers builds. Each developer works on the code on their local machines and once they have a stable addition to SINATRA, they commit it to the GitHub\textsuperscript{\textregistered} repository where it is merged into the current version. Whenever other users are working, they can pull those changes on to their local machine. This helps development move along smoothly, although it does require communication between the developers to ensure they are not working on the same lines of code and creating different outcomes. This commit structure also allows the code base to be version controlled, which helps with mistakes, reverting to older versions, and following the change logs. \par


\begin{figure}
\includegraphics[width=.99\textwidth]{figures/github-flow.png}
\centering
\caption[Example of GitHub\textsuperscript{\textregistered} workflow]{Example of GitHub\textsuperscript{\textregistered} workflow \cite{github}}
\label{fig:github}
\end{figure}




\indent Another feature of GitHub\textsuperscript{\textregistered} utilized during the development of SINATRA was the branches feature. Branches allow a user to create a separate branch of the code base. An example workflow can be seen in Figure \ref{fig:github}. As shown, there is a master branch which holds the current official version. When a developer wants to build a new feature, they create a branch of the code. On that branch they develop the feature so that the master can continue to be used for official uses. Once the feature is complete the developer submits a pull request which shows the feature and resulting changes to the master branch. The team then approves the feature and it is merged into the official master branch. Branches are used to develop a large new section of code with the features and security of GitHub\textsuperscript{\textregistered} and without cluttering the team's main code with testing and validation edits. This process allows developers to command their own section of the code, update and test, and then make it available for the other developers to use. This ensures a constant work-flow where multiple developers can work on different features without having to worry about whether their testing and tinkering will hinder others' work on the master branch. In this way, the development continues without bottlenecks, or need for constant communication between the developers. \par


\indent The final feature used in GitHub\textsuperscript{\textregistered} were the ReadME files. These files are automatically displayed by GitHub\textsuperscript{\textregistered} when the directory is entered. This allows the developers to convey how a directory fits into the code base, specifics on the files in the directory, and instructions on how to use the directory. The author has outfitted all of SINATRA's directories with ReadME files. These features were the reason that the author choose GitHub\textsuperscript{\textregistered} to host SINATRA, and they were utilized during development in order to allow efficient and clear code creation. SINATRA's main ReadME page is shown in Figure \ref{fig:readme}, as shown on the GitHub\textsuperscript{\textregistered} website. 

\begin{figure}
\includegraphics[width=.95\textwidth]{figures/Readme_redacted.png}
\centering
\caption{SINATRA's main ReadME page}
\label{fig:readme}
\end{figure}


\subsection{Doxygen}
Doxygen is an automatic documentation creator from source code.\footnote{Doxygen:  Main Page - \url{http://www.doxygen.nl/}} It was chosen, configured, and used to create the SINATRA Developer Manual. Doxygen is given access to the source code, which it searches through and creates comprehensive documentation of the code base. For SINATRA, it creates descriptions of all of the classes, functions, and files. It shows what each class consists of. For example, it shows that the Mesh class contains structure classes, public types, public member functions and more as seen in Figure \ref{fig:Doxygen_Mesh}. It is built in HTML so each attribute is linked to the actual function. It is also possible to see the location of that item in the source code. \par

% to do talk about how simple it is to comment

\indent The important part of Doxygen is that it allows the user to customize the documentation. It allows the HTML file to be built in many different ways to make it work best for SINATRA. But more importantly, it takes comments made in the source code about the attributes and displays them in a clear and concise way. This allows a developer to quickly find an attribute where it is referenced in the rest of the source code, where it is built, and what the creator commented about it. This allows new developers to quickly learn SINATRA and start developing their own features. It also allows quick debugging of heritage code by new developers, which ensures that they will not run into an error that can only be reasonably fixed by the original developer. The author has created a Doxygen manual of SINATRA, as well as a Doxygen input file and batch script for easily updating the manual as more developers add to SINATRA. See Appendix \ref{app:doxygenlists} for a list of all classes and files in SINATRA which have been commented with Doxygen. This is not a comprehensive guide to exactly what each function and variable does, but the author has commented every function in SINATRA to guide new developers. 


\begin{figure}
\includegraphics[width=.95\textwidth]{Doxygen_Mesh.png}
\centering
\caption{Documentation created by Doxygen for the Mesh Class}
\label{fig:Doxygen_Mesh}
\end{figure}


\subsection{User Distribution}

As mentioned above, GitHub\textsuperscript{\textregistered} will be the primary source of distribution for developers of SINATRA. However, for less serious users, there is a simple distribution. It includes the mesh and simulation executables and input files. It also includes the GUI executable and finally a simple MATLAB\textsuperscript{\textregistered} analysis script to visualize the particles. It has documentation which guides the user through a simple test case and show the user how the parts work together. This walk through can be found in Appendix \ref{app:walkthrough}. This distribution is packaged up in a zip file. The executables and other parts will only work on a Windows computer. Mac and Linux capability can be built through the source code and compiled on a Mac OS and Linux machine respectively. \par 


\indent This zip file can be sent to students for them to try simple test cases and produce new results. This distribution, while very simple and light in terms of file count, is actually a nearly full distribution of SINATRA. Almost all functions of the SINATRA work without the rest of the file structure and other code, therefore, it can be used for things such as uploading to a more powerful computer to ensure the correct version of the code is being used. This allows a more direct way of controlling the simulation in new environments; however, it does exclude much of the functionality found within the actual code base itself. There are many test cases and options which can be activated in the source code. This distribution can be used to share SINATRA with relevant parties who want to produce DSMC results.

\section{Workflow}
There are three main tasks when using CFD: meshing, simulation, and analysis. DSMC-PIC is a subsection of CFD; therefore, the original developers have designed systems for all three sections. However, most CFD analysis does not require the user to do all three tasks each time. A single mesh can produce many different simulations, and there can be many different analysis tasks from the same simulation. Therefore, SINATRA was broken down into three parts according to the three tasks it is able to accomplish. Each of those sections have their own repository, source code, and executables. They also share a resources repository between them. This allows the user to work in the meshing system and create the mesh or meshes necessary for their task. Then they move to the simulation system and use the newly created meshes as part of the simulation input. Once they have simulated the domain, they can take the created output files and run different analysis codes or create their own for their specific task.


% TO DO make sure that  cart3d has a 3d octree mesher
% TO DO - do I need trademark things here?

\begin{figure}
\includegraphics[width=.95\textwidth]{figures/UserWorkFlow.png}
\centering
\caption{Workflow System for SINATRA}
\label{fig:UserWorkFlow}
\end{figure}

\subsection{Mesh}
The original developers have designed this workflow for SINATRA while including the option for third party software to be used for the meshing and analysis sections. As seen in Figure \ref{fig:UserWorkFlow}, Cart3D\textsuperscript{TM} \footnote{Cart3D Documentation - \url{http://www.nanda.org/}} was chosen to be the meshing 3rd party additional tool. Meshing a domain for SINATRA requires an octree mesh. This capability is available within native SINATRA itself. The homegrown meshing tool is able to create an octree mesh with any user specified resolution. When an object is added to the domain, meshing becomes a much more complicated task (outside the scope and purpose of SINATRA). Cart3D\textsuperscript{TM} is a CFD analysis tool by NASA which is available to universities. It has within it a three-dimensional octree meshing tool, which can take geometries and domain sizing as inputs. SINATRA will take the outputted mesh and use it for simulations. Cart3D has not been tested with SINATRA. It has been slated for future work for another developer to fully integrate Cart3D\textsuperscript{TM} and geometry boundaries with SINATRA\footnote{It may be necessary to run Cart3D within a Linux\textsuperscript{TM} virtual box for Windows\textsuperscript{TM} users}. This thesis uses the homegrown meshing tool exclusively. \par

\subsection{Simulation}

SINATRA has been designed to be simple to develop and execute. Execution is completed through one executable file and one input text file. For simplicity, Windows\textsuperscript{TM} users can drag the .txt file onto the .exe file to run the simulation. It can also be run through the command line with the path to an input file as the command line argument, or if the argument is missing SINATRA will request the user to input the filepath. The executable and output do not depend on using a specific Integrated Developer Environment like Visual Studio\textsuperscript{TM} or even using a certain operating system A user can run many simulations or string together meshing, simulating, and analysis simply through a batch script. An example script is shown in Appendix \ref{app:examplescript}. \par
\indent SINATRA was also deliberately built to be machine independent to reduce the risk of the code not being developed further or used for new tasks. This was accomplished through making compilation very simple. It requires only a single command with no additional libraries\footnote{Need OpenMP\textsuperscript{TM} for parallelization}. There are sample compile statements in the ReadME and compile scripts, but even an intermediate C++ developer could figure out how to compile SINATRA from the file list alone. This helps new developers move quickly through the code learning phase and can even allow beginners to explore the code base and test more complicated features. SINATRA has been tested through being compiled with various compilers and on different operating systems. 

% TO DO add ppt graphic of this codeflow one with all the files and titles - appendix


\subsection{Analysis}
% TO DO make analysis functions which gooes through all the cells (send anamynous function)


 After the simulation phase is completed, the user can use the SINATRA output files for analysis. The original developers created MATLAB\textsuperscript{\textregistered} scripts within SINATRA to perform basic types of analysis. For other analysis, Tecplot 360\textsuperscript{TM} \footnote{Tecplot 360 CFD post processing tools to analyzedata - \url{https://www.tecplot.com/products/tecplot-360/}} has been chosen as a third party tool. Tecplot\textsuperscript{TM} is a specific CFD analysis and visualization tool, which can show the mesh, geometry, and fluid flow. It includes robust visualization and animation tools as well as various analysis functions. SINATRA can output data in a format that Tecplot\textsuperscript{TM} reads natively. Tecplot\textsuperscript{TM} and SINATRA's integration has been tested and used by the first developers.\par
 \indent SINATRA's analysis section has not been built with an encompassing set of features to complete any task. It is up to the future users to determine the analysis they need to accomplish, edit SINATRA's output class to accommodate, and compile the output data into the format best suited the situation. This can be completed through looking at SINATRA's output class and reformatting other analysis techniques for the task at hand. Tecplot\textsuperscript{TM} and the included MATLAB\textsuperscript{\textregistered} scripts can do a majority of the beginning analysis, but the most detailed tools will need to be built by new developers.

 
\subsection{Execution Time}
\label{sec:execution}
A DSMC code is by nature a very computationally intense program. It requires a large amount of memory to store all of the data of each particle and mesh cell. It requires a lot of computational power to calculate the movement and collisions, therefore, it is important to design the code to be efficient and powerful. This is why C++ and classes were chosen for SINATRA; however, it is still a slow simulation for a large meshes with high particle densities. It is slated as future work for a developer who specializes in computer science and computational optimization to reduce the execution time of large SINATRA runs. However, there are a few bottlenecks which were removed by the author. This improvement helps manage simulation time, especially when the Poisson equation solver is included. \par

\indent The simplest and most effective way to reduce simulation time on a DSMC simulation is parallelization. Parallelization is a complex and involved field with many competing ideas on best practices. There are many discussions about best ways to parallelize DSMC codes and PIC codes. Parallelization itself is also on the forefront of new technology at the time of this thesis. Moore's law has allowed programmers to have a large amount of memory for their simulation, so that is rarely the constricting factor. Processors seem to be approaching an asymptote in terms of their power for user made systems on languages like C++. Breaking the simulation between multiple cores or even within the graphics processing unit (GPU) seems to be the new normal for decreasing execution times. For SINATRA, there are many parallelization possibilities. It is slated for future work for another developer to optimize the parallelization capacity. At this time, simple parallelization has been developed by the author. During the particle propagation phase, there is no interaction between the commands, therefore, it can be parallelized by using the library OpenMP\textsuperscript{TM} \footnote{Home - OpenMP - \url{https://www.openmp.org/}}. This was completed through first identifying the loop in the simulation which moves each particle. Then the loop is reformatted to remove dependencies on any variables which must be updated sequentially. This allows the simulation to run in many different cores without interfering with the other cores. Finally, debugging is preformed to find any errors like segmentation faults and pointer errors. Parallization has been implemented into the input file. It can be enabled through an optional keyword in the input file and ensuring the library is included in the compile statement. \par


\begin{figure}
\includegraphics[width=.55\textwidth]{figures/HPC_cluster.png}
\centering
\caption[System Architecture of Cal Poly HPC Cluster]{System Architecture of Cal Poly HPC Cluster \textmd{\cite{hpc}}}
\label{fig:hpccluser}
\end{figure}


\indent The Aerospace Department at Cal Poly, SLO owns a server that is available to aerospace students and faculty. It is referred to as Bishop High Performing Cluster \cite{hpc}. It includes a workload manager so that each user only needs to submit their jobs and they are separated through the different nodes on the cluster. The cluster allows each full user 48 cores and 64 GB of Ram \cite{hpc}. This is critical to SINATRA. Through the parallelization the simulation can be run in parallel on the server. Not only does Bishop have quick processors, but it also has the 48 cores between which the simulation can be broken up. This enables the execution time of a large simulation to be cut down by significant amounts, seen in Table \ref{tab:Timing}. This server also allows users to run simulations off of their local machines, which helps compartmentalize heavy simulation runs from everyday tasks. It also allows scheduling of multiple simulations such that large analysis tasks can be completed without human interface. SINATRA has been made with the Bishop cluster as an expected resource. \par

\indent Another simple process to reduce the execution time is during the linking phase. After the particles are created, they must be associated with the cells that they are in. To do this it requires a nested for loop  through all particles inside all cells. This is a very slow process. In order to increase the speed, it is possible to allow the linking process to ignore the particles it has already linked and ones which aren't in the domain. This has been been implemented by the author, however, there is a simpler solution. There is an option in SINATRA to seed particles in a uniformly random way \cite{Galvez2018a}. This method creates the same number of particles in each cell but randomly distributes them within the cell. With uniform initialization, the linking process can be removed completely, and the execution time is significantly reduced. This has been implemented by the author. This was completed through reformatting how particles are created in SINATRA. A new constructor for particles was created which includes the index of the cell which it was created in as well as the index of the next particle for the particle chain. The other constructors for particles are updated so that they explicitly state that they are not yet linked to a cell and don't know their next particle. These are two main things are what are added to the particle object during linking. Then during initialization, the cells are given the information of their first particle and the total number of particles in this cell. These two things are what are added to the cells during linking. These two changes allow linking to be removed for the uniform initialization case. \par

\begin{table}
\caption{Execution Time Comparisons}
\label{tab:Timing}
\vspace{0.3cm}
\begin{center}
\begin{tabular}{|l|l|l|l|}
\hline
                             & First Iteration & Time-step Average & Total Time     \\ \hline
Parallel and No Linking & 9 sec           & 10 sec           & 1 min, 38 sec  \\ \hline
Serial and No Linking   & 3 min, 23 sec   & 3 min, 23 sec    & 33 min, 50 sec \\ \hline
Serial and Linking      & 15 min, 1 sec   &  3 min, 58 sec   &   50 min, 47 sec             \\ \hline
\end{tabular}
\end{center}
\end{table}

\indent A simple timing comparison was completed to examine the effectiveness of two execution time reduction methods. A collision-less simulation with all diffuse walls was run on the HPC server. The simulation used 32768 Cells, \(5 \times 10^{19}\) real to simulated particles, \(1 \times 10^{26}\) number density which gives 2 million simulation particles in the domain, and 10 time-steps of \(1 \times 10^{-8}\) seconds with 0 m/s stream velocity and 300 Kelvin temperature. As seen in Table \ref{tab:Timing}, including these optimization techniques has reduced the simulation time by 49 minutes for this simulation. By removing the linking process, we were able to save a large amount of time during the set up\footnote{There is a slight difference in the iteration time for the linking and no linking runs because the no linking run did not use the uniform initialization because uniform initialization has has the linking section bypassed in the current version.}. Also, by adding the simple parallelization we were able to reduce the simulation time by 95\%, from over 50 minutes to under 2 minutes. These two techniques combined make an appreciable difference in the execution time. They allow more complicated simulations to be run within the same amount of computation time. This allows the user to run longer simulations or multiple in order to increase confidence in the accuracy of the solution, as well as reduce the variance caused by the randomness in a DSMC simulation. \par

\indent Finally, an important tool for execution time analysis is debugging and profiling. It is important while developing software to have the ability to debug the code to see exactly what it is doing. This is usually achieved through the Integrated development environment (IDE); however, in an effort to make SINATRA IDE independent, the code has built in a method which allows the GNU Project Debugger to be used on the code\footnote{GDB: The GNU Project Debugger - \url{https://www.gnu.org/software/gdb/}}. This is a command line interface for g++ compiled executables. It is a legacy debugger that is well documented and is included in most installations of the g++ compiler. Using gdb on SINATRA allowed the author to solve problems within various improvements including removing linking and adding parallelized code. Secondly, a profiler allows the user to view exactly how often each line of code is run, how long it takes to run, and the sub processes, which contributes to that execution time. This is a strong way to identify simple coding bottlenecks and optimizing the coding time. The author has set up the compiling and code base in order to fit with the profilier Very Sleepy\textsuperscript{TM} \footnote{Very Sleepy documentation - \url{http://www.codersnotes.com/sleepy/}}. This was completed by determining the version of g++ which the profiler required, compiling in that method, and then allowing the profiler to access the executable. Very Sleepy\textsuperscript{TM} is a light and simple profiler which shows each line of the source code and the timing involved. By setting up SINATRA to be widely compatible it allowed these developing tools to push SINATRA towards an industry standard level. \par


% Cite processer information
% Cite DSMC paralization papers


\input{sections/UserInterfaces}

\chapter{Charged Particles}
A DSMC simulation has many applications in aerospace engineering. However, it does not capture the full picture when gasses are composed of ions and electrons. Plasma, which is composes of charged particles, needs the electrostatics to be taken into account in the simulation in order to create an accurate model. This chapter will explain how to simulate charged particles with the particle in cell method, how that method was implemented in SINATRA, and validation of the implementation.
\label{chap:charge}
\section{Particle In Cell}

As said before, plasma is quasi-neutral in that it has about even densities of ions and electrons. However, there can be patches of charge inequality. Therefore, the gradient between the particles creates an electric field, which in turn applies a force on the particles and changes their velocities. \par


\subsection{Description of Equations}

The characteristic length of those charge inequalities can be calculated through the Debye Length, \(\lambda_{De}\), shown in Equation \ref{eqn:debye} \cite{debye}. \par


\Needspace{8\baselineskip}
\begin{equation}
    \label{eqn:debye}
    \lambda_{De} = \sqrt{\frac{\epsilon_0 \: T_e}{e \: n_0}}
\end{equation}
\(T_e\) = Electron Temperature \\
\(\epsilon_0\) = Permittivity of free space \\
\(e\) = elementary charge \\
\(n_0\) = plasma density \par

\indent This Debye length is the first plasma property which helps with setting up the simulation. The Debye length gives us the a good approximation for cell size when creating a PIC mesh. \par 

\indent Similar to the Molecular Dynamics method, it is possible to model every particle in a domain \footnote{This explination of the Electro-Static PIC method draws much of it's material from \cite{es-pic} because it was used as a reference and it is attempting to explain the same method at the same level as this thesis.}. Charged interactions between particles would be calculated through the Coulomb Force, seen in Equation \ref{eqn:coulmb}. However, as seen by the \(q_1\) and \(q_2\) in the Coulomb force, the force on each particle would have to be the sum of every other particle in the domain. This is still computationally in-feasible. \par 

\Needspace{8\baselineskip}
\begin{equation}
    \label{eqn:coulmb}
    \vec{F} = \frac{1}{4 \pi \epsilon_0} \frac{q_1 q_2}{r^2}  \: \vec{r}_{12}
\end{equation}
\(\epsilon_0\) = Permittivity of Free Space \\
\(q\) = Charge of a particle \\
\(r\) = Distance between particle 1 and 2 \\
\(\vec{r}_{12}\) = The direction of the line connecting the two particles \par

\indent In PIC, the particles move according to the Lorenz force, seen in Equation \ref{eqn:lorenz}. This force can be calculated across the entire domain and then applied to each particle, reducing the order of the simulation to \(O(n)\) \cite{es-pic}. \par

\Needspace{8\baselineskip}
\begin{equation}
    \label{eqn:lorenz}
    \vec{F} = q (\vec{E} + \vec{v}  \times \vec{B})
\end{equation}
\(q\) = Particle Charge \\
\(E\) = Electric Field \\
\(v\) = Particle Velocity \\
\(B\) = Magnetic Field \par

\indent It is not in the scope of this thesis to examine the effect of the self induced or an applied magnetic field, therefore the force is reduced to \(\vec{F} = q \vec{E}\). The electric field can be calculated through the gradient of the electric potential, as shown in Equation \ref{eqn:e_field}. \par


\Needspace{5\baselineskip}
\begin{equation}
    \label{eqn:e_field}
    \vec{E} = - \nabla \phi
\end{equation}
\(\phi\) = Electric Potential \par

\indent The electric potential comes from the double gradient of the charge density. The electric potential can be seen in Equation \ref{eqn:poisson}, which is also known as the Poisson Equation, specifically for electrostatics. \par

\Needspace{5\baselineskip}
\begin{equation}
    \label{eqn:poisson}
    \nabla^2 \phi = - \frac{\rho}{\epsilon_0}
\end{equation}
\(\rho\) = Charge Density \\
\(\epsilon_0\) = Permittivity of Free Space \par

\indent The charge density is found in Equation \ref{eqn:density}. More information on the PIC method to calculate the charge density is given in section ref{the section}. \par

\Needspace{5\baselineskip}
\begin{equation}
    \label{eqn:density}
    \rho = e(Z_i n_i - n_e)
\end{equation}
\(Z_i\) = Ion Charge \\
\(n\) = number density \\
\(i\) = subscript for Ions \\
\(e\) = subscript for electrons \par

\indent These set of equations give us the particle in cell method. Calculate the charge density, then electric potential, then the electric field. Use that field to update the velocities of the particles. The addition of these calculations into the DSMC method is shown in Figure \ref{fig:pic_flow}. However, an important assumption is made for this thesis. Electrons are much smaller than ions and therefore travel much faster. The difference in speeds between the two means that a simulation which includes both ions and electrons as simulated particles would need to have a time-step which captured the electron movement, and therefore would be wasting a lot of time on the ions barely moving. In order to simplify the simulation, a fluid assumption is made for the electrons. This is a valid assumption when viewed from the ions reference frame because the electron particles themselves move so fast that the ions are only affected by the overall distribution of electrons. This fluid assumption is calculated through the Boltzmann relationship \cite{es-pic}, and can be seen in Figure \ref{eqn:e_density}.


\Needspace{5\baselineskip}
\begin{equation}
    \label{eqn:e_density}
    n_e = n_0 \exp{\frac{\phi - \phi_0}{k T_e}}
\end{equation}
\(n\) = number density \\
\(\phi\) = Electric Potential \\
\(\phi_0\) = Initial Electric Potential \\
\(T_e\) = Temperature of the Electrons \\
\(k\) = Boltzmann Constant \par


% example from here http://www.texample.net/tikz/examples/simple-flow-chart/

\subsection{Particle in Cell Algorithm}
\label{sec:algorithm}

\tikzstyle{block} = [rectangle, draw, fill=white, 
    text width=5em, text centered, rounded corners, minimum height=4em]
\tikzstyle{block_pic} = [rectangle, draw, fill=red!20, 
    text width=5em, text centered, rounded corners, minimum height=4em]
\tikzstyle{line} = [draw, -latex']

\begin{figure}
\centering
  \begin{tikzpicture}[node distance = 2cm, auto]
    % Place nodes
        \node [block] (init) {Initialize System};
        \node [block_pic, below of=init] (density) {Calculate Charge Density};
        \node [block_pic, below of=density] (poisson) {Solve Poisson's Equation};
        \node [block_pic, below of=poisson] (e_field) {Calculate Electric Field};
        \node [block_pic, left of=init, node distance=3cm] (velocity) {Update Particle Velocity};
        \node [block, below of=e_field] (stop) {Stop Simulation};
        % Draw edges
        \path [line] (init) -- (density);
        \path [line] (init) -- (velocity);
        \path [line] (density) -- (poisson);
        \path [line] (poisson) -- (e_field);
        \path [line] (e_field) -- (stop);
    \end{tikzpicture}
    \label{fig:pic_flow}
    \caption{DSMC-PIC Hybrid Code Flow. White blocks are DSMC algorithms \cite{Galvez2018a}. Red are PIC. IN WORK - WAITING ON TEX FILES FROM DAVID}
\end{figure}

There are many ways to take these formulas and code flow and turn into a working simulation.  There could be multiple methods for all of these calculations, which leads to the various techniques for PIC codes, similar to the various DSMC codes. The ones chosen for this are simple and robust. First, while the DSMC method is based upon cells full of particles, the PIC method depends on the nodes of the cells upon which the fields can be calculated. When calculating the charge density, the charge must be distributed to those nodes. This is done through weighted charge distribution. The distance between the particle and the 8 nodes of it's cell is calculated, and then the 8 different volumes which the particle cuts the cell into are calculated. Those weights determine the amount of charge from the particle distributed to each node. \par

\indent Once this is done, Poisson's equation can be solved. This is the most computationally intensive part of a PIC code because Poisson's equation is a partial differential equation. To solve it in the PIC framework, it is discredited across the nodes. Then the different discretized partial differential equation solvers can be utilized to calculate the electric potential. For SINATRA, the Finite Difference method is used, as discussed in Section \ref{sec:finite_diff}. In order to solve the Finite Difference a linear equation solver must be implemented, which in the case of SINATRA is Gauss-Seidel. This implementation is shown in Section \ref{sec:gauss}. \par

\indent Once the electric potential solver is shown to be converged, the electric field can be calculated. A popular way to do this, and the implementation in SINATRA, is using a forward difference method. This can be seen in Equation \ref{eqn:forward}. This equation is only for the x direction and not for along a boundary. \par

\Needspace{5\baselineskip}
\begin{equation}
    \label{eqn:forward}
    E_{x,i} = - \frac{\phi_{i+1,j,k} - \phi_{i-1,j,k}}{2 \Delta x}
\end{equation}
\(x\) = Distance between nodes in the X direction \\
\(\phi\) = Electric Potential \\
\(i \: \text{,} \: j \: \text{and} \: k\) = The x,y, and z node directions respectively  \par

\indent These three steps only need to be calculated once during the timestep, hence the beauty of the PIC method. During the Move Particle part of the DSMC code, these calculations are utilized. Before updating the position of the particle, the velocity is updated. Rearranging and simplifying Equation \ref{eqn:lorenz} gives the acceleration as \(a_x = \frac{q}{m} E_x\). This is used to update the velocity before updating the position of the particle. Then all that is left to do is incorporate PIC settings into the input and output systems of the DSMC and a functioning DSMC-PIC code is born. 
% change this if do the leap frog method. 









\section{PIC Implementation}

Implementing PIC into SINATRA was completed through following the algorithm seen in Section \ref{sec:algorithm}. Two important factors were critical during implementation: robustness and efficiency. SINATRA is built to be able to simulate many different fluid scenarios chosen by the user. Therefore, the PIC portion should also have this generality. Therefore, even though the Finite Difference method requires equal cell sizes, the rest of the algorithms are based on the individual cells so that when the mesh is non-uniform the simulation could be updated with minimum work. It also was implemented in an efficient manner for the DSMC setup and algorithms found in SINATRA. This means it used flattened arrays, used the built in particle and cell index arrays, and allowed for easy parallelization. 

\subsection{Finite Difference}
\label{sec:finite_diff}

\indent Finite Difference was chosen as the method to discrete Poisson's equation. Finite difference is a simple and common method for discrediting the Poisson Equation and is common in PIC codes. There are other algorithms which could be used, but finite difference allows for a relatively simple implementation. The derivation below is for a 7 point finite difference for the Poisson equation \cite{FD_GS} \cite{FDM}.

Starting with the Poisson equation expanded out to each of the 3 directions. 

\begin{equation}
    \label{eqn:poisson_expanded}
    \nabla^2 \phi = \frac{\partial^2 \phi}{\partial x^2} + \frac{\partial^2 \phi}{\partial y^2} + \frac{\partial^2 \phi}{\partial z^2} = - \frac{\rho}{\epsilon_0}
\end{equation}

% Make those 
Next, assuming discretization upon nodes designated by \(i \: \text{,} \: j \: \text{and} \: k\) representing nodes in the \(x \: \text{,} \: y \: \text{and} \: z\) directions respectively \cite{FD_GS}. 

\begin{equation}
    \label{eqn:x_partial}
    \frac{\partial^2 \phi}{\partial x^2}(x_i,y_i,z_i) \approx \frac{1}{h^2}(\phi(x_{i-1},y_j,z_k) - 2\phi(x_i,y_j,z_k) + \phi(x_{i+1},y_j,z_k)),
\end{equation}
\begin{equation}
    \label{eqn:y_partial}
    \frac{\partial^2 \phi}{\partial x^2}(x_i,y_i,z_i) \approx \frac{1}{h^2}(\phi(x_i,y_{j-1},z_k) - 2\phi(x_i,y_j,z_k) + \phi(x_i,y_{j+1},z_k)),
\end{equation}
\begin{equation}
    \label{eqn:z_partial}
    \frac{\partial^2 \phi}{\partial x^2}(x_i,y_i,z_i) \approx \frac{1}{h^2}(\phi(x_i,y_j,z_{k-1}) - 2\phi(x_i,y_j,z_k) + \phi(x_i,y_j,z_{k+1})),
\end{equation}

Plugging these into Equation \ref{eqn:poisson_expanded} gives.

\begin{align}\label{eqn:poisson_full}
    \frac{\phi(x_{i-1},y_j,z_k) - 2\phi(x_i,y_j,z_k) + \phi(x_{i+1},y_j,z_k)}{h^2} +& \nonumber \\
    \frac{\phi(x_i,y_{j-1},z_k) - 2\phi(x_i,y_j,z_k) + \phi(x_i,y_{j+1},z_k)}{h^2} +& \nonumber \\
    \frac{\phi(x_i,y_j,z_{k-1}) - 2\phi(x_i,y_j,z_k) + \phi(x_i,y_j,z_{k+1})}{h^2} =& - \frac{\rho}{\epsilon_0}
\end{align}

This is a set of linear equations, which lends itself nicely to set up a \(A x = b\) situation. This is a common problem in linear algebra and therefore multiple methods exist to solve it. First, the matrix must be created. This matrix, also called a stencil, only needs to be created once per simulation because it is only dependent on the mesh purely. Equation \ref{eqn:stencil} shows the stencil for a 7 point mesh on a 3x3x3 node domain. \par

% Make this prettier  \ddots
\begin{equation}
\label{eqn:stencil}
A = 
\begin{bmatrix}
S & I &  & I &  &  &  &  & \\ 
I & S &I  &  & I &  &  &  & \\ 
 & I & S &  &  &I  &  &  & \\ 
I &  &  & S & I &  &I &  & \\ 
 & I &  & I & S & I &  &I  & \\ 
 &  & I &  & I & S &  &  & I\\ 
 &  &  & I &  &  & S & I & \\ 
 &  &  &  & I &  & I & S & I\\ 
 &  &  &  &  & I &  & I & S
\end{bmatrix}
\end{equation}
where,
\begin{equation} \nonumber
S = 
\begin{bmatrix}
-6 & 1 & \\
1 & -6 & 1\\
 & 1 & -6 \\
\end{bmatrix}
\ \text{and} \ I = 
\begin{bmatrix}
1 &  & \\
 & 1 & \\
 & & 1 \\
\end{bmatrix}
\end{equation}

\indent This stencil was implemented into SINATRA as a flattened 2D matrix which in itself two flattened 3D matrices. The electric potential is stored as a flattened 3D matrix. The matrix items are selected through conversion functions that go from 3 dimensions to 1 dimension and vice verses. Figures \ref{fig:sparse} shows a Matlab "spy" command on SINATRA's stencil. The spy command shows which items in a matrix are non-zero. As seen by these figures the stencil is largely empty. Future work would include using a different data structure and access algorithm to reduce the wasted memory.  \par


\begin{figure}
    \centering
  \begin{minipage}[b]{0.49\textwidth}
    \includegraphics[width=\textwidth]{figures/sparse_8.jpg}
  \end{minipage} %
  \begin{minipage}[b]{0.49\textwidth}
    \includegraphics[width=\textwidth]{figures/sparse_64.jpg}

  \end{minipage}
  \caption[Sparse Stencil Matrix]{SINATRA Stencil Matrices, shown through Matlab\textsuperscript{TM}'s spy command (left) 8 Cells in the mesh (right) 64 cells in the mesh.}
  \label{fig:sparse}
\end{figure}


\indent At this stage, the simulation assumes equal cell sizes and refinement across the whole domain. However, the charge density summation is calculated cell by cell, and the electric field is distributed cell by cell. These allow future iterations to have unequal cell sizes. \par


\subsection{Gauss-Seidel}
\label{sec:gauss}

The Gauss-Seidel iterative method was chosen to be the linear algebra solver for SINATRA. It was chosen for it's simplicity, universality, and robustness. It only has one improvement over the simplest iterative method, Jacobi. This thesis will not go into a derivation of the Gauss-Seidel method, but the equation to update the electric potential at each timestep is given by Equation \ref{eqn:gauss_seidel}.

\begin{equation}
    \label{eqn:gauss_seidel}
    x_i^{k+1} = ( b_i - \sum_{j=1}^{i-1} a_{i j} x_j^{k+1} - \sum_{j=i+1}^n a_{i j} x_j^{k} ) / a_{ii}
\end{equation}

In terms of SINATRA,
\(x\) = Electric Potential \\
\(b\) = Charge Density + Boltzman Relationship \\
\(a\) = Stencil \par


\indent The Charge Density as defined in Equation \ref{eqn:density} involves the electron density, which is assumed to be a fluid currently. It was discusses how to distribute the ion's charge, but not how to combine the ion charge and the electron charge. Therefore, by combining Equations \ref{eqn:poisson}, \ref{eqn:density}, and \ref{eqn:e_density} and then discretizing the left hand side of Poisson's Equation, \ref{eqn:poisson} and putting in matrix form, \ref{eqn:stencil}, we get Equation \ref{eqn:mixed}.

\Needspace{5\baselineskip}
\begin{equation}
    \label{eqn:mixed}
    A x = - \frac{e}{\epsilon_0} [n_i - n_o exp(\frac{\phi - \phi_0}{k \: T_e})]
\end{equation}
\(n\) = number density \\
\(\phi\) = Electric Potential \\
\(\phi_0\) = Initial Electric Potential \\
\(T_e\) = Temperature of the Electrons \\
\(k\) = Boltzmann Constant \par

% capital right hand side?
\indent This shows that the \(b\) in Equation \ref{eqn:gauss_seidel} is the right hand side of the set of linear equations the Gauss-Seidel solver will solve. Importantly note that the right hand side depends on the electric potential, or \(x\) as seen on the left hand side. \par

\section{Results}


\subsection{Solver Validation}


\subsection{Test Cases}



\subsection{Validation}

\chapter{Conclusions}
\label{chap:conclusions}

The goal of this thesis is three-fold. The first goal is to help develop a homegrown DSMC simulation tool for Cal Poly. The second goal is to implement charged particles into the DSMC to create a hybrid DSMC-PIC simulation. The third goal is to design the code architecture and systems to have both a low probability of becoming unusable and a similarity to the upcoming homegrown fluid simulation. These are all intermediary goals which work towards the final end of creating a full simulation of an electric engine. This thesis shows that the three goals have been accomplished through the latest upgrade of the new homegrown code, SINATRA. \par

\indent Building a homegrown CFD code is a complex task. The early developers have worked together to build a robust and flexible DSMC code base. This has been demonstrated through Alliston and Galvez's theses \cite{Galvez2018a,mac_thesis}. It has the ability to model many different flow conditions, species, and boundary conditions. Its execution time (i.e. under 2 minutes for 2 million particles) allows for accurate simulations in an institution research setting. It has been validated across its various collision and sphere models. \par


\indent In order to eventually model a full electric thruster, this part of the simulation must be able to model charged particles for the plume. This has been achieved by making SINATRA a DSMC-PIC hybrid. A simple and robust method is used for discretizing Poisson's Equation: The Finite Difference method. This involves creating a 7-point 3D stencil, Neumann boundary conditions, slip conditions through averaging. For solving the resulting set of linear equations, the Gauss-Seidel iterative method was chosen. The implementation in SINATRA was verified against a verified solver and shown to converge on the 3D stencil. The accuracy of the PIC implementation has been validated through two important test cases, ambipolar diffusion and steady state flow. In the ambipolar diffusion test case, it is clear that the charged particles create a force which greatly reduces diffusion so that within \(5\times 10^{-5}\) seconds only 0.071\% of the particles leave the domain. With the steady state flow, the potential was seen to quickly assume a steady state solution and change slowly as the particles moved through the domain. \par

\indent The code base has been converted from a small developer's program to a university standard code. It is managed through GitHub\textsuperscript{\textregistered} in order to be used by multiple university developers. It has been upgraded with a user manual, consistent and simple documentation, and a simplified workflow. Two large bottlenecks, particle pushing and particle linking, have been greatly mitigated through parallization and increased data throughput so that timely simulations may be made. In addition, a distribution method has been created with easy-to-use executables and a GUI for user control. \par

\indent There are still many ways to improve upon SINATRA. Those are discussed in Section \ref{sec:future}. However, SINATRA is a strong first iteration of a university level DSMC-PIC code base. This work will help SINATRA reach its goals and Cal Poly will enabled to do novel research in modeling full electric thrusters.

\section{Future Work}
\subsection{Electric Thruster}
\subsection{SINATRA}
\subsection{Charged Particles}